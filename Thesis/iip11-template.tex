\documentclass[twoside]{article}
\usepackage{iip11}

\begin{document}


\title{Шаблон оформления тезисов доклада на~конференцию ИОИ"=11}
\author{Автор~И.\,О.}{Автор Имя Отчество$^{1, 2}$}{author\_email@site.ru}
\author{Соавтор~И.\,О.}{Соавтор Имя Отчество$^2$\speaker}{coauthor@site.ru}
\organization{%
    $^1$Город, Организация\par
    $^2$Город, Организация}
\Russian
\maketitle

Сборник тезисов докладов будет издан к~началу конференции.
Тезисы доклада должны занимать ровно одну страницу на~русском языке.
Перевод на~английский язык обязателен,
он~должен находиться в~том же файле и~также занимать одну страницу
(одна страница приблизительно соответствует 1800 символам).
Программный комитет оставляет за собой право сократить текст тезисов,
если он не умещается на одной странице.

Для~каждого автора указывается фамилия с~инициалами,
фамилия с~именем и~отчеством полностью,
адрес электронной почты.
Номер автора для указания аффилированных организаций ставится после отчества без пробела.
Если авторов несколько, то желательно отметить командой \verb|\speaker| (\speaker) того автора, который будет выступать с~докладом.
Тезисы доклада могут содержать формулы, таблицы, иллюстрации.
Тезисы доклада не должны содержать разделов, списков, сносок, плавающих иллюстраций.
Ссылку на грант(ы) можно указать в последней строке.

Список литературы должен содержать ровно один пункт~---
ссылку на полную версию данной статьи,
принятую в~печать в~российском или международном рецензируемом журнале,
либо поданную в~рецензируемый электронный журнал <<Машинное обучение и анализ данных>>
через сайт \url{http://jmlda.org}.
Для электронных журналов URL статьи указывается обязательно.
Тезисы доклада могут дословно повторять аннотацию полной версии статьи, но могут и~дополнять её.
Название тезисов доклада может совпадать с~названием статьи.
Решение о~принятии или отклонении доклада на~конференцию ИОИ"=11
будет приниматься Программным комитетом на~основании текста статьи.

Работа поддержана грантом РФФИ \No\,00-00-00000.

\begin{thebibliography}{1}
\bibitem{author16first-word-of-the-title}
    \emph{Автор\;И.\,О.}
    Название статьи~//
    Название журнала,
    Город: Издательство, 2016.~--- С.\,5--25. %\\
    \url{http://jmlda.org/papers/doc/2016/no1/Author2016Title.pdf}.
\end{thebibliography}


\title{Abstract template~--- IIP"=11}
\author{Author~N.}{Author Name$^{1,2}$}{author\_email@site.ru}
\author{Coauthor~N.}{Coauthor Name$^2$\speaker}{coauthor@site.ru}
\organization{%
    $^1$City, Institution\par
    $^2$City, Institution}
\English
\maketitle

The abstract should be concise and should present the aim of the work, essential results and conclusion.
The total length of the abstract should not exceed one page, that is approximately 1800 characters.
The abstract may contain plots, formulas and tables, in case if it helps to illustrate the results.
The abstract should not contain sections, lists, footnotes or floating images.
A reference to the grant can be provided in the last line.

The title should include full names, e-mails and institutional affiliations for all authors. The speaker should be marked using the command \verb|\speaker| (\speaker).
The bibliography must contain exactly one item~--- a reference to the full version of the article,
published or accepted for publication in a peer-reviewed journal,
or submitted to the peer-reviewed electronic journal~``Machine Learning and Data Analysis''
through the website of the journal \url{http://jmlda.org}.
If the full version of the article is published in an electronic journal, the URL for this article must be indicated.

This research is funded by RFBR, grant 00-00-00000.

\begin{thebibliography}{1}
\bibitem{author16first-word-of-the-title}
    \emph{Author\;N.}
    Paper name~//
    Journal,
    City:~Publisher, 2016.~--- p.\,5--25.
    \url{http://jmlda.org/papers/doc/2016/no6/Author2016Title.pdf}.
\end{thebibliography}

\end{document}
