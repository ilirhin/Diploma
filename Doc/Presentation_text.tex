\documentclass[12pt]{article}
\usepackage[utf8]{inputenc}
\usepackage[russian]{babel}
\usepackage{amsmath}
\usepackage{amssymb}
\usepackage{MnSymbol}
\usepackage{wasysym}
\usepackage{mathtext}
\usepackage{mathenv}
\usepackage{listings}
\usepackage{color}
\usepackage{graphicx}
\usepackage{hyperref}
\usepackage{amssymb,amsfonts,amsmath,mathtext,cite,enumerate,float}
%\usepackage{algorithm}
\usepackage{float}
\usepackage[noend]{algpseudocode}
\usepackage[ruled,vlined]{algorithm2e}
\usepackage[a4paper, left=25mm, right=20mm, top=20mm, bottom=20mm]{geometry}
\usepackage{indentfirst}
\usepackage{epstopdf}

\DeclareGraphicsExtensions{.eps} 


\newtheorem{definition}{Определение}[section]
\newtheorem{remark}{Примечание}[subsection]
\newtheorem{suggest}[remark]{Соглашение}
\newtheorem{claim}[remark]{Утверждение}
\newtheorem{lemma}[remark]{Лемма}
\newtheorem{theorem}{Теорема}
\newtheorem{conseq}{Следствие}[theorem]
\newenvironment{Proof} 
	{\par\noindent{\bf Доказательство.}} 
	{\hfill$\blacksquare$}

\newenvironment{rusalgorithm}[1][htb]
  {\renewcommand{\algorithmcfname}{Алгоритм}
   \begin{algorithm}[#1]
  }{\end{algorithm}}



\definecolor{dkgreen}{rgb}{0,0.6,0}
\definecolor{gray}{rgb}{0.5,0.5,0.5}
\definecolor{mauve}{rgb}{0.58,0,0.82}

\lstset{frame=tb,
  language=Java,
  aboveskip=3mm,
  belowskip=3mm,
  showstringspaces=false,
  columns=flexible,
  basicstyle={\small\ttfamily},
  numbers=none,
  numberstyle=\tiny\color{gray},
  keywordstyle=\color{blue},
  commentstyle=\color{dkgreen},
  stringstyle=\color{mauve},
  breaklines=true,
  breakatwhitespace=true
  tabsize=3
}

\renewcommand{\baselinestretch}{1.4}
\renewcommand{\leq}{\leqslant}
\renewcommand{\geq}{\geqslant}
\renewcommand{\phi}{\varphi}

\DeclareMathOperator{\Supp}{Supp}

\newcommand{\norm}{\mathop{\mathsf{norm}}\limits}
\newcommand{\sparse}{\mathop{\mathsf{sparse}}\limits}
\newcommand{\argmin}{\mathop{\mathsf{argmin}}\limits}
\newcommand{\argmax}{\mathop{\mathsf{argmax}}\limits}

\begin{document}
\paragraph{Титульный слайд}
Здравствуйте. Меня зовут Ирхин Илья, мой научный руководитель --- д.ф.-м.н. Воронцов Константин Вячеславович. Тема дипломной работы --- сходимость численных методов вероятностного тематического моделирования. Перейдём к плану презентации.

\paragraph{Краткое содержание}
В начале будет сформулирована задача вероятностного тематического моделирования. Затем будет рассказано о подходе аддитивной регуляризации тематических моделей и рассказаны цели проведённой работы.

Потом будут изложены полученные теоретические результаты, в рамках которых были предложены условия сходимости и предложены модификации алгоритма, улучшающие эту сходимость. В конце будут показаны результаты проведённого эксперимента, в котором проверяось выполние условий сходимости и сравнивались предложенные модификации.

\paragraph{Тематическое моделирование текстовых коллекций}
Итак, перейдём к модели тематического моделирования. Имеется коллекция документов, каждый документ состоит из слов. Для каждого документа известны частоты слов, которые в нём встречаются. Предполагается, что появление слов в документах обусловлено некоторыми ненаблюдаемыми величинами --- темами, то есть сначала определяется тема документа, а затем эта тема генерирует слово. Тогда вероятность встретить слово в документа можно факторизовать по темам и получить формулу, которую вы видите на слайде.

Чтобы найти параметры $\phi_{wt}$ и $\theta_{td}$ предлагается решать задачу максимизации правдоподобия. 

\paragraph{Подход ARTM}
Однако эта оптимизационная задача крайне неустойчива, поскольку у функции правдоподобия очень много локальных максимумов. Также крайне важно какое начальное приближение используется при оптимизации. Поэтому предлагается ввести регуляризатор и максимизировать сумму логарифма правдоподобия и регуляризатора.

Применяя теорему Каруша-Куна-Такера, получается система уравнений на переменные  $\phi_{wt}$ и $\theta_{td}$ . Решение данной системы методом простых итераций даёт ЕМ алгоритм.

\paragraph{Подход ARTM}

Е-шаг состоит в нахождении внутренних оценок на вероятность темы каждого слова в каждом документе.

на М-шаге строятся оценки для мат.ожидания количества встреч слов в темах и тем в документах, находятся регуляризационные поправки и выполняет нормировка с положительной срезкой.

\paragraph{Цели работы}
На практике алгоритм ARTM сходится, однако, нет теоретических гарантий для этого. Поэтому первостепенная цель работы было изучение свойств сходимости алгоритма ARTM.

Нужно было найти условия на регуляризаторы, которые легко проверять аналитически или можно обеспечить при реализации, которые будут гарантировать сходимость

Проведя анализ алгоритма, предложить модификации, улучшающие его сходимость.

Проведение эксперимента для проверки выполнения полученных условий, а также сравнения предложенных модификаций и стандартного алгоритма ARTM.

\paragraph{Теорема}
Алгоритм ARTM можно проинтерпретировать как обобщённый ЕМ алгоритм с априорным распределением на параметры. Это даёт возможность перенести результаты о сходимости GEM алгоритмов, которые хорошо изучены, на ARTM.

Позвольте объяснить смысл предложенных условий.

Первые два означают, что если параметры занулились на некоторой итерации процесса, то они будут оставаться нулевыми на всех последующих итерациях. Его легко проверить аналитически.

Третье условие это отделимость от нуля параметров модели. Позволяет учесть машинную точность, также можно в реализации алгоритма занулять все параметры меньшие $\varepsilon$. 

Четвёртое условие --- это невырожденность правдоподобия. Оно означает, что правдоподобие ненулевое на итерациях. Обеспечивается за счёт коэффициента регуляризации.

Пятое и шестое условия --- это невырожденность М-шага. Там производится нормировка, данное условие требует чтообы знаменатель был ненулевой, а при $\delta > 0$ отделён от нуля.

Седьмое условие --- общее для GEM алгоритмов, на каждой итерации вводится нижняя оценка на регуляризованное правдоподобие и требуется неуменьшать данную функцию.

При выполнение данных условий можно утверждать, что разность между параметрами на соседних итерациях стремится к нулю.

\paragraph{Следствия}
Приведены некоторые следствия из теоремы, предлагаю вернуться к ним в конце презентации, если будут вопросы.

\paragraph{Несмещённая модификация}
Для доказательства условия №7 (увеличение функции нижней оценки на каждой итерации) предлагается заменить все вхождения  $\phi_{wt}$ и $\theta_{td}$ на их несмещённые оценки.

\paragraph{Градиентная модификация}
Также была предложена другая модификация --- использование градиента регуляризатора в качестве регуляризационных поправок (выделены красным).

Это выражение можно эффективно вычислить, поскольку второе слагаемое одникаковое для всех слов (тем) в рамках одной темы (документа).

\paragraph{Обоснования формул}
Можно найти градиент $R$ по $n_{wt}$ и $n_{td}$. Для несмещённой модицикации угол между изменением и градиеннтом всегда острый а для градиентной модификации угол нулевой. Поэтому на М-шаге будет просиходить локальное увеличение.


\paragraph{Обоснования формул подробнее}
На этом слайде краткий набросок доказательства того, что для несмещённой модификации угол с градиентом острый. Предлагаю пока пропустить этот слайд.

\paragraph{Различия}
На этом слайде показаны различия предложенных формул на примере простого регуляризатора, опять же не будем углубляться.

\paragraph{Эксперимент}
Перейдём непосредственно к проведённому эксперименту.
\begin{enumerate}
\item В качестве регуляризатора был взят регуляризатор декоррелирования ($R = -\tau\sum\limits_w \sum\limits_{t \neq s} \phi_{wt} \phi_{ws}$).
\item Были проверены четыре случая величины $\tau$: $10^5,~10^6,~10^7,~10^8$, для разного числа тем: 3, 10, 30.
\item Использовались статьи со спортивного сайта sports.ru по 7 видам спорта.
\item Проверялись стандартная формула, замена несмещёнными оценками и градиентное преобразование.
\item Алгоритм  запускался из случайных начальных приближений, одинаковых для всех запусков, после чего сравнивались средние значения целевых метрик.
\end{enumerate}

\paragraph{Графики}

\paragraph{Итоги}
Подведём итоги эксперимента.
\begin{enumerate}
\item Предположения об отделимости и невырожденности распределений выполняются на практике.
\item С точки зрения оптимизации $L +  R$ все рассмотренные формулы отличаются незначительно. Основное достоинтсво предложенных модификаций в этом плане --- это теоретические гарантии и более эффективная оптимизация $R$.
\item Есть эффект скачков значений функционалов на первых итерациях для стандартной формулы, вызванный выбором точки в которой считаются $r_{wt}$ и $r_{td}$.
\item Для градиентных поправок необходимо дополнительное исследование, чтобы понять как подбирать константу перед градиентом.
\end{enumerate}

\paragraph{Результаты выносимые на защиту}
В заключение приведём результаты выносимые на защиту.
\begin{enumerate}
\item Условия для сходимости ЕМ-алгоритма ARTM, легко проверяемые и обеспечиваемые при реализациии.
\item Две модификации формул М-шага ЕМ-алгоритма, улучшающие сходимость без увеличения вычислительной сложности.
\item Оценки изменения значений регуляризатора и логарифма правдоподобия на итерациях для предложенных модификаций.
\end{enumerate}
Спасибо за внимание.

\end{document}